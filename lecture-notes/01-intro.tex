% Imports
\documentclass[12pt]{article}
\usepackage[margin=1in]{geometry}
\usepackage{
    amsmath, amsthm, amssymb, verbatim
}

% Begin document
\setlength{\parindent}{0pt}
\begin{document}

% Title
\title{Applied Econometrics I Workshop}
\author{Arturo Soberon}
\date{}
\maketitle

% Intro to Causal Inference
\setcounter{section}{0}
\section{Introduction}
Throughout the course, we will be interested in estimating the causal effect that some
variable $X$ has on some outcome $Y$. This is a common concern among lab scientists,
software developers, politicians, etc. Some people are super lucky and can estimate the
causal effect of $X$ on $Y$ in a lab setting. Unfortunately, we economists usually
cannot rely on experiments to discover causal relationships due to ethical or budget
constraints.

For example, if we were interested in measuring the causal effect that building a
hospital has on the health of people who live nearby, we would face the following
problems:
\begin{enumerate}
    \item Hospitals arguably improve people's health, and thus it would be unethical to
    randomly choose where to build new hospitals. This could misallocate public
    resources by favoring communities that already have good healthcare access over
    underserved ones.
    \item Hospitals are expensive to build, so we cannot justify a multi-billion
    dollar experiment where we randomly pick where to build new hospitals and hire
    health professionals.
\end{enumerate}

As a consequence, we economists have to get creative to measure causal effects. This
course is dedicated to learning creative designs that allow us to derive causal insights
from non-experimental data. In other words, we will learn how to use data drawn from
real life and use it to draw causal insights.

Formally, we are interested in estimating the causal effect of a treatment $D$ on an
outcome $Y$. Suppose the treatment is binary and let $i$ be an observed unit (person,
lab rat, etc.). The outcome for unit $i$, $Y_i$, can be expressed in terms of its
\textit{potential outcomes}:

\begin{equation}
    Y_i =
    \begin{cases}
        Y_{0i}, & \text{if } D_i = 0 \text{ (unit is not treated)} \\
        Y_{1i}, & \text{if } D_i = 1 \text{ (unit is treated)}
    \end{cases}
\end{equation}

In a perfect testing environment, we could observe $Y_{1i}$ for unit $i$, rewind time,
and then observe $Y_{0i}$ for the same unit. This way we could measure the true causal
effect of the treatment on $i$ ($\tau_i = Y_{1i} - Y_{0i})$.

Unfortunately, in the real world, we can only observe one of the two potential
outcomes for each unit. This is the fundamental problem of causal inference. Given this
limitation, a natural starting point is to compare the average outcomes of the treated
and control groups.

\begin{equation}
    E[Y_i | D_i = 1] - E[Y_i | D_i = 0]
\end{equation}

However, this simple difference combines two distinct elements. We can decompose this
term to see what is really going on:

\begin{equation}\label{decomp}
\begin{split}
    E[Y_i | D_i = 1] - E[Y_i | D_i = 0] & = E[Y_{1i} | D_i = 1] - E[Y_{0i} | D_i = 0] \\
    & = E[Y_{1i} | D_i = 1] - E[Y_{0i} | D_i = 0] \\
    & \text{\qquad} + E[Y_{0i} | D_i = 1] -
    E[Y_{0i} | D_i = 1] \\
    & = \underbrace{E[Y_{1i} - Y_{0i} | D_i = 1]}_{\text{ATET}} + \underbrace{
        E[Y_{0i} | D_i = 1] - E[Y_{0i} | D_i = 0]
    }_{\text{Selection Bias}}
\end{split}
\end{equation}

This equation implies that whenever we take the difference in means between two groups,
the observed value is the sum of two terms:
\begin{enumerate}
    \item The average causal effect of $D$ on $Y$ for the \textbf{treatment
    group}. This is known as the Average Treatment Effect on the Treated (ATET).
    \item The average outcome the treatment group \textit{would have had} if they had
    not been treated, minus the average outcome of the control group. We call this term
    \textbf{Selection Bias}.
\end{enumerate}

There are two main issues here. First, we are not estimating the Average Treatment
Effect (ATE) for the whole population. If anything, we are estimating the ATET, but it
is \textit{contaminated} by the second term (selection bias).

The easiest way to fix this is by randomly assigning the treatment. Doing this allows
you, the researcher, to design an experiment where you can randomly assign the
treatment to the units in the population. Like we mentioned earlier, this is seldom the
case in Economics, but Randomized Controlled Trials (RCTs) are an important thing to
learn for anyone interested in causal inference.

So, assume $D_i$ is randomly assigned. Mathematically, this means that the potential
outcomes are independent of the treatment assignment. This forces the selection bias
term to zero:
\[
E[Y_{0i} | D_i = 1] - E[Y_{0i} | D_i = 0] = E[Y_{0i}] - E[Y_{0i}] = 0
\]
Under this assumption, we can simplify equation \eqref{decomp}. The simple difference
in means now isolates the causal effect:
\begin{equation}
    E[Y_i | D_i = 1] - E[Y_i | D_i = 0] = E[Y_{1i} - Y_{0i} | D_i = 1] = E[Y_{1i} -
    Y_{0i}]
\end{equation}

In other words, when the treatment is randomly assigned, the observed difference in
average outcomes is a valid estimate of the Average Treatment Effect.

% End document
\end{document}
